\documentclass{article}

%package setup
\usepackage{graphicx}
\usepackage{amsmath}
\usepackage{fancyhdr}
\usepackage[margin=1in]{geometry}
\usepackage{comment}
\usepackage{placeins}
\usepackage{parskip}
\usepackage{subcaption}
\usepackage{appendix}
\usepackage{soul}
\usepackage{comment}
\PassOptionsToPackage{hyphens}{url}\usepackage[hidelinks]{hyperref}
\usepackage{matlab-prettifier}
\usepackage{minted}
\usepackage{enumitem}
\usepackage{float}
\usepackage{textcomp, gensymb}
\usepackage{caption}


\pagestyle{fancy}
\fancyhf{} % Clear header/footer settings
\rhead{\thepage} % Page number on the right in the header
\lhead{ASE375 Lab Report 9} % Your lab report title on the left

\begin{document}

\begin{titlepage}
  \centering
  \includegraphics[width=10cm]{ase-logo-formal.png}  % Adjust the width as needed
  \vspace{1cm}  % Add some vertical space
 
  \Large \textbf{ASE 375 Electromechanical Systems}\\
  \large \textbf{Section 14115}\\
  \vspace{0.5cm}
  \textbf{Monday: 3:00 - 6:00 pm}\\
 
  \vspace{1cm}
 
  \hrule
  \vspace{0.5cm}
 
  \Huge \textbf{Report 9:\\
    Digital Image Correlation}\\
  \Huge \textbf{}\\
 
  \vspace{0.5cm}
  \hrule
 
  \vspace{1cm}
 
  \normalsize \textbf{Andrew Doty, Andres Suniaga, Dennis Hom}\\
  \normalsize \textbf{Due Date: 04/22/2024}
 
\end{titlepage}
\newpage

\tableofcontents
\thispagestyle{empty}
\newpage

\section{Introduction}
In this experiment we investigate the stress concentration around the center holes of (1) an Aluminum and (2) Poly-carbonate test article by using a universal testing machine (UTM) to apply a tensile force to these test materials until failure. Using Digital Image Correlation (DIC) we are able to measure the strain distribution around the center hole of each test article.
\vspace{2.5mm}

This experiment will involve a comparison between our collected stress concentration values and the calculated theoretical values. This lab provides insight into methods used in materials science to determine the strength of materials, such as the stress concentration factor of our Aluminum and Poly-carbonate strips.

\section{Equipment}
Measurement devices and hardware used in this lab include:
\begin{itemize}

\item Instron 3300 Series Universal Testing Machine \hyperlink{1}{[1]} and Bluehill software: 
\vspace{1mm}

Machine used to apply a tensile load to the test articles. Bluehill software used to interface with the testing machine to apply a tensile load to our test articles.
\begin{figure}[H]
    \centering
    \includegraphics[width = 0.6\textwidth]{lab9images/lab9_emech_setup.png}
    \caption{Experiment Setup}
    \label{fig:setup9}
\end{figure}
\vspace{2.5mm}

\item Digital Calipers, $\epsilon_{b} = 0.005\; \text{mm}$: 
\vspace{1mm}

Used for measuring outer and inner dimensions of objects. In our case we use it to the measure the diameter of the center hole of test articles along with the strips' thickness and width before the experiment.
\vspace{2.5mm}

\item Aluminum and Poly-carbonate Strips:
\vspace{1mm}

Material strips measured in this lab. Used as shown in Figure \ref{fig:setup9}. Material properties of these strips can be found in Table \ref{tab:materialproperties}.

\vspace{2.5mm}

\item Web Camera and Digital Image Correlation Engine (DICe) software \hyperlink{2}{[2]}:
\vspace{1mm}

Used to capture images through a sequence of tensile loads applied to the test articles, as shown in Figure \ref{fig:setup9}. DICe used to analyze strain distribution of the set of images.
\vspace{2.5mm}
\end{itemize}

\section{Procedure}
\subsection{Tensile Loading}
\begin{enumerate}
    \item Before beginning the experiment, measure the thickness, width, and center hole diameter of the test articles with the digital calipers.
    \item Turn on the UTM and open up the Instron Bluehill Software:
    \begin{figure}[H]
        \centering
        \includegraphics[width = 0.7\textwidth]{lab9images/instrom_bluehill_page.PNG}
        \caption{Bluehill UI}
        \label{fig:bluehill}
    \end{figure}

    Load in the appropriate configuration file for the test articles to the Bluehill software.

    \item Secure one of the test articles within the Instron machine as shown in Figure \ref{fig:setup9}. Ensure that the test article is tightly in place. 

    \item Open the web camera application and ensure the camera is focused on the strip. Once focused, take a snapshot of the strip before loading. Now, begin the loading for the strip in the machine. 
    \begin{itemize}
        \item For Aluminum, take snapshots for every 1000N load (or 1kN)
        \item For Poly-carbonate, take snapshots for every 100N load.
    \end{itemize}

    \item The loading will end once the strips have reached failure. Take a snapshot of the broken strip. 
    \item Save these images. Repeat loading for the other test strip.
    \item Next we will move onto the Digital Image Correlation Process.
\end{enumerate}

\subsection{Digital Image Correlation}
\begin{enumerate}
\item Open up the DICe software:
\begin{figure}[H]
    \centering
    \includegraphics[width = 0.7\textwidth]{lab9images/DICE_software_page.PNG}
    \caption{DICe UI}
    \label{fig:dice}
\end{figure}

\item Load in the reference image (before tensile loading). Then load in the sequence of images after the reference once the strip begins to deform.

\item Use the line tool to draw a vertical line on the reference image as close to the edge of the center hole as possible. This will output a plot once the 2D images are analyzed.

\item On the right side panel, set the contour field to VSG\_STRAIN\_YY. Now on the left side panel, select 'run 2d'.
\begin{itemize}
    \item NOTE: Increasing the SSIG threshold may help with the analysis. The SSIG threshold was set to the max for our analysis.
\end{itemize}

 \item Run through the sequence of images analyzed with the 'frames' slider. There will be a plot of the strain distribution along the line drawn from earlier. 

 \item For each test article, select a good image and plot that represents the strain distribution around the center hole.

 \item This concludes the procedure. Processing this data and comparing to the theoretical values is next.

\end{enumerate}
\hypertarget{datapro}{}
\section{Data Processing}
\subsection{Variables and Equations}  

Nominal Stress at a cross-section containing the hole diameter:
\begin{equation}
    \sigma_{\text{nom}} = \dfrac{P}{t(D-d)}
\end{equation}

Variables:
\begin{itemize}
    \item \(P\): Applied tensile load
    \item \(t\): Thickness of test strip
    \item \(D\): Width of the test strip
    \item \(d\): Center hole diameter
\end{itemize}
\vspace{5mm}

Stress Concentration factor:
\begin{equation}
    K_{t} = \dfrac{\sigma_{\text{max}}}{\sigma_{\text{nom}}}
\end{equation}

Variables:
\begin{itemize}
    \item \(\sigma_{\text{max}}\): Maximum stress
\end{itemize}
\vspace{5mm}

Theoretical Stress Concentration factor (for $0 \leq d \leq D$):
\begin{equation}
    K_{t} = 3 - \pi\left(\dfrac{d}{D}\right) + \frac{11}{3}\left(\dfrac{d}{D}\right)^{2} - 1.527\left(\dfrac{d}{D}\right)^{3} 
\end{equation}
\vspace{5mm}

\begin{table}[H]
    \centering
    \begin{tabular}{c c c}
        \hline
        \hline 
        
         & Aluminum 7075-T6 & Poly-carbonate  \\  
         
         \hline
         
        Elastic Modulus & 71.7 GPa (10,400 ksi) & Avg. 2 GPa (290 ksi) \\[2pt]
        Yield Strength & 503 MPa (73 ksi) & 62.0 MPa (8.99 ksi) \\[2pt]
        Approx. Yield Strain & 0.7\% & 7.0\% \\[2pt]
        Poissons Ratio & 0.33 & 0.37 \\[2pt]
        
        \hline
        \hline
    \end{tabular}
    \caption{Material Properties of Test Articles}
    \label{tab:materialproperties}
\end{table}


\section{Results and Analysis}

%-----------------------------------------------------------------%
%Organize images of the aluminum and polycarbonate in this section%
%-----------------------------------------------------------------%

The aluminum and polycarbonate test strips were subjected to tensile loading using the Instron 3300 Series Universal Testing Machine until failure. Digital Image Correlation (DIC) was employed to measure the strain distribution around the center hole of each test article. Figure 4 shows the aluminum strip at the point of breakage, while Figure 5 depicts the polycarbonate strip at failure. The failure patterns differ between the two materials due to their varying material properties.

\begin{figure}[H]
    \centering
    \includegraphics[width = 0.6\textwidth]{lab9images/aluminum_break.jpg}
    \caption{Aluminum at breakage}
    \label{fig:alfail}
\end{figure}

\begin{figure}[H]
    \centering
    \includegraphics[width = 0.6\textwidth]{lab9images/polycarbonate_break.jpg}
    \caption{Poly-carbonate at breakage}
    \label{fig:PCfail}
\end{figure}

DIC analysis was performed on the sequence of images captured during the tensile loading process. Figure 6 presents the strain distribution image for the aluminum strip at a load of 8 kN. The color plot clearly shows the concentration of strain around the center hole, with the highest strain values (indicated by the red and yellow regions) located near the hole's edge. Figure 7 provides the corresponding strain distribution plot along a vertical line drawn near the hole's edge. The peak strain value reaches approximately 0.15 at the hole's edge, demonstrating the stress concentration effect.

\begin{figure}[H]
    \centering
    \includegraphics[width = 0.6\textwidth]{lab9images/meow color plot aluminum.PNG}
    \caption{Aluminum strain distribution image at 8kN}
    \label{fig:al8kNpic}
\end{figure}



\begin{figure}[H]
    \centering
    \includegraphics[width = 0.6\textwidth]{lab9images/8kNplot_Al_strainyy.png}
    \caption{Aluminum strain distribution plot at 8kN}
    \label{fig:al8kNplot}
\end{figure}



\begin{figure}[H]
    \centering
    \includegraphics[width = 0.6\textwidth]{lab9images/1800Npccolored_strainyy.PNG}
    \caption{Poly-carbonate strain distribution image at 1800N}
    \label{fig:PC1800Npic}
\end{figure}

Similarly, Figure 8 displays the strain distribution image for the polycarbonate strip at a load of 1800 N. The color plot reveals a more diffuse strain distribution compared to the aluminum strip, with the highest strain values concentrated around the center hole. Figure 9 shows the strain distribution plot for the polycarbonate strip, with a peak strain value of about 0.035 near the hole's edge.

\begin{figure}[H]
    \centering
    \includegraphics[width = 0.6\textwidth]{lab9images/1800Npcplot_strainyy.png}
    \caption{Poly-carbonate strain distribution plot at 1800N}
    \label{fig:PC1800Nplot}
\end{figure}

The force-displacement curves for both the aluminum and polycarbonate strips, as measured by the Instron machine, are presented in Figure 10. The aluminum strip exhibits a linear elastic behavior followed by a sudden failure, while the polycarbonate strip shows a more ductile response with a larger displacement before failure. This difference in behavior can be attributed to the inherent material properties of the two test articles, as shown in Table 2.

\begin{figure}[H]
    \centering
    \includegraphics[width = 0.6\textwidth]{lab9images/forcevsdisplacement.png}
    \caption{Force vs Displacement on Instron Machine for Polycarbonate and Aluminum}
    \label{fig:forcedisplacement}
\end{figure}

\begin{figure}[H]
    \centering
    \includegraphics[width = 0.6\textwidth]{lab9images/strainvsforce.png}
    \caption{Force vs Strain on Instron Machine for Polycarbonate and Aluminum}
    \label{fig:strainforce}
\end{figure}

\begin{table}[H]
    \centering
    \begin{tabular}{|l|c|c|}
    \hline
     & \textbf{Aluminum} & \textbf{PolyCarbonate} \\ \hline
    Inner Diameter (d) & \(9.5\pm0.05\) & \(9.44\pm0.05\) \\ \hline
    Outer Diameter (mm) & \(25.21\pm0.05\) & \(25.18\pm0.05\) \\ \hline
    Thickness (mm) & \(1.4\pm0.05\) & \(2.79\pm0.05\) \\ \hline
    \(\sigma_{\text{MAX}}\) (Mpa) & \(367.55255\) & \(58.92364157\) \\ \hline
    Elastic Modulus (Gpa) & \(71.7\) & \(2\) \\ \hline
    Strain at \(\sigma_{\text{MAX}}\) & \(0.148\) & \(0.09\) \\ \hline
    \(K_t\) (Measured) & \(2.891\) & \(3.103\) \\ \hline
    \(K_t\) (theoretical) & \(2.256\pm0.07\) & \(2.258\pm0.06\) \\ \hline
    \end{tabular}
    \caption{Comparison of Aluminum and PolyCarbonate properties}
    \label{table:materials}
\end{table}
    
The measured stress concentration factors (Kt) for the aluminum and polycarbonate strips are 2.891 and 3.103, respectively. These values are higher than the theoretical Kt values of 2.256 ± 0.07 and 2.258 ± 0.06, calculated using the provided equation. The discrepancy between the measured and theoretical values may be due to factors such as material imperfections, machining irregularities, or limitations in the DIC analysis.
 
\section{Conclusion}

In this experiment, the stress concentration effect around the center holes of aluminum and polycarbonate test strips was investigated using tensile loading and Digital Image Correlation (DIC). The results demonstrate that the presence of a hole in the test articles leads to a localized increase in strain, with the highest strain values observed near the hole's edge. The aluminum strip exhibited a brittle failure mode, while the polycarbonate strip showed a more ductile behavior, as evidenced by their respective force-displacement curves and failure patterns. \\

The measured stress concentration factors (Kt) for both materials were found to be higher than the theoretical values, indicating the influence of factors not accounted for in the theoretical calculations. This discrepancy highlights the importance of experimental verification in determining the actual stress concentration effects in materials. \\

The DIC technique proved to be a valuable tool for visualizing and quantifying the strain distribution around the center holes, providing insights into the local deformation behavior of the test articles. The results of this experiment emphasize the significance of considering stress concentration effects in the design and analysis of components containing discontinuities, such as holes, to ensure their structural integrity and performance.

\newpage
\thispagestyle{empty}  % Clear header/footer
\begin{center}
	\vspace*{\fill}
	{\Huge Appendix}
	\vspace*{\fill}
\end{center}

% Start appendices
\newpage
\begin{appendices}
\pagestyle{fancy}
\renewcommand{\thefigure}{A\arabic{figure}}
\setcounter{figure}{0}

\pagebreak

\hypertarget{datasheets}{}
\section{Datasheets}
\begin{enumerate}[label = {[\arabic*]}]
\small
\item \hypertarget{1}{\href{https://www.instron.com/en/products/testing-systems/universal-testing-systems/low-force-universal-testing-systems/3300-series}{Instron 3300 Series Table Operator UTM}}
\item \hypertarget{2}{\href{https://www.sandia.gov/ccr/software/digital-image-correlation-engine-dice/}{Digital Image Correlation Engine (DICe) Software}}


\end{enumerate}

\end{appendices}

\end{document}
