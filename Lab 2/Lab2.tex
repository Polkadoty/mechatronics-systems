\documentclass{article}
\usepackage{graphicx}
\usepackage{amsmath}
\usepackage{fancyhdr}
\usepackage[margin=1in]{geometry}
\usepackage{comment}
\usepackage{placeins}
\usepackage{parskip}
\usepackage{subcaption}
\usepackage{appendix}
\usepackage{soul}
\usepackage{comment}
\usepackage[hidelinks]{hyperref}
\usepackage{matlab-prettifier}
\usepackage{minted}
\usepackage{enumitem}

\pagestyle{fancy}
\fancyhf{} % Clear header/footer settings
\rhead{\thepage} % Page number on the right in the header
\lhead{ASE375 Lab Report 2} % Your lab report title on the left

\begin{document}

\begin{titlepage}
  \centering
  \includegraphics[width=10cm]{ase-logo-formal.png}  % Adjust the width as needed
  \vspace{1cm}  % Add some vertical space
 
  \Large \textbf{ASE 375 Electromechanical Systems}\\
  \large \textbf{Section 14115}\\
  \vspace{0.5cm}
  \textbf{Monday: 3:00 - 6:00 pm}\\
 
  \vspace{1cm}
 
  \hrule
  \vspace{0.5cm}
 
  \Huge \textbf{Report 2:\\
  Temperature Sensor Measurements}\\
  \Huge \textbf{}\\
 
  \vspace{0.5cm}
  \hrule
 
  \vspace{1cm}
 
  \normalsize \textbf{Andrew Doty, Andres Suniaga, Dennis Hom}\\
  \normalsize \textbf{Due Date: 02/12/2024}
 
\end{titlepage}
\newpage

\tableofcontents
\thispagestyle{empty}
\newpage



\section{Introduction}
This experiment consisted of measuring temperature with three different sensors: a Thermocouple, Thermistor, and an Integrated Circuit Temperature sensor. Data collection was made possible through a Data Acquisition (DAQ) system used to process the different temperature measurements in LabVIEW, a graphical interface that modeled the temperature sensors' measurements in real-time. 

The purpose of this experiment was to learn how to simulate our data through LabVIEW along with observing and understanding the behaviour of the three temperature sensors in different environments: $(1)$ at room temperature, $(2)$ in water near freezing conditions, and $(3)$ in water closer to boiling conditions. 

\section{Equipment}
The equipment used in this experiment include .....

K-type Thermocouple:  connected to NI 9211

Thermistor:  connected to NI 9215 via breadboard w/ $1\text{k}\Omega$ resistor etc....

IC Temperature Sensor: connected to NI 9215

Breadboard: 

Circuit Components:  various male-to-male jumper wires, $1\text{k}\Omega$ resistor, etc... and 5V power supply

DAQ:  Data Aquisition system that digitizes analog information into "bins" for a computer.  The specific DAQ had two units, the NI 9215 and NI 9211.  Specific Datasheets for each are included in the appendices.  

Thermometer:  used to measure true temperature w/ 0.5 degrees 

Water:  Access to water near a boiling temperature, and water in an ice bath.   




\section{Procedure}
\begin{figure}[H]
\centering
\includegraphics[width=0.5\textwidth, angle = -90]{Lab 2/lab2images/circuit_board_mug_and_sensors.jpg}
\caption{Temperature Sensors, Water mug, and Breadboard circuit}
\end{figure}



\section{Data Processing}
\subsection{Variables}
\begin{enumerate}[label = \roman*.]
    \item \(N = \) Number of Samples
    \item \(f = \) Sampling Frequency
    \item \(\gamma = \) Confidence Level \%
    \item \(R_{s} = \) Sensor Resistance [Ohms = $\Omega$], in this experiment it will be $1\text{k}\Omega$
\end{enumerate}

\subsection{Equations}
\begin{enumerate}[label = \Roman*.]
    \item Sample Mean: \(\bar{x} = \dfrac{1}{N}\displaystyle\sum_{i=1}^{N} x_{i}\) 
    \item Standard Deviation of the Mean: \(S_{x} = \sqrt{\displaystyle\sum_{i=1}^{N} \dfrac{(x_{i} - \bar{x})^{2}}{N-1}}\)
    \item 
\end{enumerate}
 
\begin{figure}[H]
\centering
\includegraphics[width=0.5\textwidth, angle = -90]{Lab 2/lab2images/labview_plots.jpg}
\caption{Plotting data on LabVIEW}
\end{figure}

\subsection{Part 1}


\subsection{Part 2}


\section{Results and Analysis}



\section{Conclusion}




\newpage
\thispagestyle{empty}  % Clear header/footer
\begin{center}
	\vspace*{\fill}
	{\Huge Appendices}
	\vspace*{\fill}
\end{center}

% Start appendices
\newpage
\begin{appendices}
\pagestyle{fancy}
\renewcommand{\thefigure}{A\arabic{figure}}
\setcounter{figure}{0}

\section*{Appendix: t-Distribution Tables}
\includegraphics[width=0.95\textwidth]{Lab 1/t_distribution_Table_lecture3.png}
\end{appendices}


\end{document}
