\documentclass{article}

%package setup
\usepackage{graphicx}
\usepackage{amsmath}
\usepackage{fancyhdr}
\usepackage[margin=1in]{geometry}
\usepackage{comment}
\usepackage{placeins}
\usepackage{parskip}
\usepackage{subcaption}
\usepackage{appendix}
\usepackage{soul}
\usepackage{comment}
\usepackage[hidelinks]{hyperref}
\usepackage{matlab-prettifier}
\usepackage{minted}
\usepackage{enumitem}
\usepackage{float}
\usepackage{textcomp, gensymb}
\usepackage{tabularx}


\pagestyle{fancy}
\fancyhf{} % Clear header/footer settings
\rhead{\thepage} % Page number on the right in the header
\lhead{ASE375 Lab Report 2} % Your lab report title on the left

\begin{document}

\begin{titlepage}
  \centering
  \includegraphics[width=10cm]{ase-logo-formal.png}  % Adjust the width as needed
  \vspace{1cm}  % Add some vertical space
 
  \Large \textbf{ASE 375 Electromechanical Systems}\\
  \large \textbf{Section 14115}\\
  \vspace{0.5cm}
  \textbf{Monday: 3:00 - 6:00 pm}\\
 
  \vspace{1cm}
 
  \hrule
  \vspace{0.5cm}
 
  \Huge \textbf{Report 2:\\
  Temperature Sensor Measurements}\\
  \Huge \textbf{}\\
 
  \vspace{0.5cm}
  \hrule
 
  \vspace{1cm}
 
  \normalsize \textbf{Andrew Doty, Andres Suniaga, Dennis Hom}\\
  \normalsize \textbf{Due Date: 02/12/2024}
 
\end{titlepage}
\newpage

\tableofcontents
\thispagestyle{empty}
\newpage

\section{Introduction}
This experiment consisted of measuring temperature with three different sensors: a Thermocouple, Thermistor, and an Integrated Circuit Temperature sensor. Data collection was made possible through a Data Acquisition (DAQ) system used to process the different temperature measurements in LabVIEW, a graphical interface that modeled the temperature sensors' measurements in real-time. 

The purpose of this experiment was to learn how to simulate our data through LabVIEW along with observing and understanding the behaviour of the three temperature sensors in different environments: $(1)$ at room temperature, $(2)$ in water near freezing conditions, and $(3)$ in water closer to boiling conditions. 

\section{Equipment}
The equipment used in this experiment include the following:  

K-type Thermocouple:  Temperature sensor with two different metals joined together at one end.  A K-type thermocouple uses Chromel-Alumel metals.  It will be connected to the DAQ via the NI 9211 thermocouple input module.  

SA1-TH Series Thermistor:  Temperature sensor that measures electrical resistance as a response to a change in temperature.  It is connected to the NI 9215 via breadboard in its own circuit with $1\text{k}\Omega$ resistor.
\begin{figure}[H]
    \centering
    \includegraphics[width=0.25\textwidth]{lab2images/thermistor_circuit.jpg}
    \caption{Thermistor Circuit}
\end{figure}

TMP36 Temperature Sensor: Analog low voltage sensor.  It is connected to the NI 9215 via the breadboard.  

Breadboard: a reusable solderless prototyping board used for building electronic circuits. Components are inserted into interconnected rows and columns of holes, allowing for easy and temporary assembly of circuits for testing and experimentation.

Circuit Components:  various length male-to-male jumper wires, $1\text{k}\Omega$ resistor, 5V power supply.

DAQ:  Data Aquisition system that digitizes analog information into "bins" for a computer.  The specific DAQ had two units, the NI 9215 and NI 9211.  Specific Datasheets for each are included in the appendices.  

Thermometer:  Regular mercury thermometer, using change in volume as a response to a change in temperature.  Used to measure true temperature with 0.5 degrees least count.  

Water:  Access to water at two temperatures, near boiling, and ice cold. 

\section{Procedure}
\begin{figure}[H]
\centering
\includegraphics[width=0.5\textwidth, angle = -90]{lab2images/circuit_board_mug_and_sensors.jpg}
\caption{Temperature Sensors, Water mug, and Breadboard circuit}
\end{figure}

The DAQ, with the NI 9211 and NI 9215 modules already installed, was connected to an external 5V power supply as well as to the computer.  The LabVIEW software is used to digitize the data from the temperature sensors.  To do this, the following block diagram was created.

\begin{figure}[H]
\centering
\includegraphics[width=0.5\textwidth]{lab2images/ProcessDiagram.png}
\caption{Process Diagram}
\end{figure}

\begin{figure}[H]
    \centering
    \includegraphics[width=0.5\textwidth]{lab2images/68154.jpg}
    \caption{Process Diagram screenshot}
    \end{figure}

The DAQ block had signals for each sensor, measuring at a 1000hz frequency.  The collector is initially given a size of 500,000 samples.  Each sensor from the DAQ was given its own graph using a split signal, and a "write to measure file" block was placed to save our data to an Excel file.  

Now that our digital sensing was set up, we had to create the physical circuits. The thermocouple was connected directly into the NI 9211 module.  The NI 9215 and power supply were connected to a breadboard. 
 The Thermistor was connected to a $1\text{k}\Omega$ resistor in a voltage divider configuration. The TMP36 was connected to the 5V power supply and the NI 9215 module.  This configuration is shown in figure 2.

 Additionally, for each part of this experiment, the true temperature was measured via a mercury thermometer.

\subsection{Part 1} % Subsection 1

To gather the room temperature data, the sensors were placed in an unused area in the lab room and left to sit for a few minutes without anyone moving near them or touching them.  This was to stop heat from diffusing through the rubber casing of the wire or through the air. The LabVIEW simulation was run for 3 minutes at 1000hz and the data was saved to the Excel file. The results of the simulation are shown in the figures below.

\subsection{Part 2} % Subsection 2
 
\subsubsection{Part 2a: Hot Water} % Subsubsection 1

An electric kettle was used to heat up water to near boiling.  The LabVIEW simulation was started with the sensors in the ambient environment until the graphs appeared to reach steady-state, which acts as an asymptote on the value from the sensors, and then were abruptly submerged in the boiling water.  They stayed in the water until the graph once again appeared at steady-state.  This experiment was run over 2 minutes and at a frequency of 1000hz.  The region between the two steady-states will be used to calculate the time-constant.  

\subsubsection{Part 2b: Water Cooling} % Subsubsection 2

For this part of the experiment, the collector block's size was changed to one million samples to account for the amount of time it would take for the water to cool from its initial boiling state to a safe and drinkable condition.  At a frequency of 1000hz, this gave us approximately 16 minutes of data collection.  The sensors were placed in the boiling water and the data was collected until the true temperature, measured via a thermometer, was at \(40^\circ C\), as that is the optimal temperature for safe drinking water.  

\subsubsection{Part 2c: Ice-Hot} % Subsubsection 3

Place the three sensors in a bowl of ice-cold water, wait for the graphs to reach equilibrium, and then quickly take them out of the ice water, and into the hot water, measuring until it reaches steady-state.  The region between the equilibrium in ice water and the equilibrium in hot water is the time constant we are measuring.  In our experiment, the ice water was at \(40^\circ C\) and the hot water was at \(87^\circ C\).  

\subsubsection{Part 2d: Quick Changes} % Subsubsection 4

Lastly, after the sensor had reached steady state in the water above, we took the sensors out of the water and measured the time it takes for the sensors to return to room temperature. We saved the data to an Excel file with a frequency of 1000 hz. 

\subsubsection{Part 2e: Sensor Repetition} % Subsubsection 5

The data was measured simultaneously for each sensor.  

\section{Data Processing}
\subsubsection*{Variables}
\begin{enumerate}[label = \roman*.]
    \item \(N = \) Number of Samples
    \item \(f_{s} = \) Sampling Frequency, $s^{-1}$
    \item \(\Delta t_{s} = \) Sampling Interval, $s$
    \item \(\gamma = \) Confidence Level, \%
    \item \(R_{S} = \) Sensor Resistance, Ohms = $\Omega$ (In this experiment it will be $1\text{k}\Omega$)
    \item \(V_{S} = \) Source voltage, Volts = $V$ ($5\;V$ for this experiment)
    \item \(R_{T} = \) Themistor resistance, $\Omega$
\end{enumerate}

\subsubsection*{Equations}
\begin{enumerate}[label = \Roman*.]
    \item Sample Mean: \(\bar{x} = \dfrac{1}{N}\displaystyle\sum_{i=1}^{N} x_{i}\) 
    \item Standard Deviation of finite $N$, normalized by $N-1$: \(S_{x} = \sqrt{\displaystyle\sum_{i=1}^{N} \dfrac{(x_{i} - \bar{x})^{2}}{N-1}}\)
    \item Standard Deviation of the Mean: \(\dfrac{S_{x}}{\sqrt{N}}\)
    \item Average Measurement w/ Confidence Interval: \(\bar{x} \pm t_{stat}\cdot \dfrac{S_{x}}{\sqrt{N}}\)
    \item \textit{Steinhart-Hart Relation}: \(\dfrac{1}{T} = A + B\cdot ln(R_{T}) + C\cdot (ln(R_{T}))^{3}\), where $A$, $B$, $C$ are the Thermistor's calibration coefficients.
\end{enumerate}  

\subsection*{Conversion and Calibration}
The IC/TMP36 sensor originally outputs voltage. This was converted to $\degree C$ by taking the IC TMP36 output, say $x_{V}$ ($V$ for volt), and converting by $x_{\degree C} = 100\cdot(x_{V}-0.5)$. This conversion comes from the TMP36 datasheet ($10\,mV/\degree C$ w/ $500\,mV$ offset):
\begin{center}
    \href{https://www.analog.com/media/en/technical-documentation/data-sheets/tmp35_36_37.pdf}{\includegraphics[width = 0.6\textwidth]{lab2images/tmp36_offset_documentation.png}}
\end{center}


We utilize \textbf{Eq.(V)} to calibrate the Thermistor.

Thermistor Calibration:
\[
    T = \dfrac{1}{A + B\cdot ln(R_{T}) + C\cdot (ln(R_{T}))^{3}}
\]

By gathering the Thermistor's resistance along with the temperature in each environment we can set up and solve a system of linear equations as shown below:
\begin{center}
    $\begin{bmatrix}
        1 & ln(R_{T_{1}}) & (ln(R_{T_{1}}))^{3}\\[3pt]
        1 & ln(R_{T_{2}}) & (ln(R_{T_{2}}))^{3}\\[3pt]
        1 & ln(R_{T_{3}}) & (ln(R_{T_{3}}))^{3}
    \end{bmatrix}
    \cdot
    \begin{bmatrix}
        A\\[3pt]
        B\\[3pt]
        C
    \end{bmatrix}
    = 
    \begin{bmatrix}
        \frac{1}{T_{1}+273.15}\\[3pt]
            
        \frac{1}{T_{2}+273.15}\\[3pt]
        
        \frac{1}{T_{3}+273.15}
    \end{bmatrix}$
\end{center}

Temperature is converted to Kelvin for calculation purposes. This temperature is the (approximate) true measurement. The $1$,$2$,and $3$ subscripts represent the three environments (in no particular order).


\subsubsection*{Mean Temperature in Laboratory w/ Confidence Interval}
Calculation of the mean temperature with confidence interval of $\gamma = 95\%$ was implemented through MATLAB as shown below:
\begin{lstlisting}[style=Matlab-editor]
% Confidence Interval in Laboratory
gamma = 0.95; %95 percent confidence

Fz = 0.5*(1+gamma);

nu = Q-1; %DOF, where Q is the # of samples

p = (1-gamma)/2; %probability

tstat = tinv(p,nu); 

% (SAMPLE MEANS) Mean w/ Confidence Interval
avg_unc_thermistor = [mean(thermistordata) 
                     tstat*std(thermistordata)/sqrt(Q)];

avg_unc_IC = [mean(ICdata) 
             tstat*std(ICdata)/sqrt(Q)];

avg_unc_thermocouple = [mean(thermocoupledata)  
                       tstat*std(thermocoupledata)/sqrt(Q)];
\end{lstlisting}

We define $F(z) = \frac{1}{2}(1+\gamma) = 0.9750$. The degrees of freedom (D.O.F.) $\nu = N - 1 = 59,999$ where $N=60,000$ samples. Set $P=\dfrac{1-\gamma}{2}=0.0250$. Using MATLAB function \textit{tinv}, we calculate the t-statistic to be $t_{stat} = -1.960$ or according to the \hyperlink{1}{t-distribution tables} this is also just $t_{stat}=1.960$.

Below are the average temperatures w/ the confidence interval for each of the three sensors:
\begin{itemize}
    \item Thermistor: \(1.1743 \pm 0.0001\; V \)
    \item IC/TMP36: \(20.8415 \pm 0.0038\; \degree C\)
    \item Thermocouple: \(21.6671 \pm 0.0005\; \degree C\)
\end{itemize}


\section{Results}
This section analyzes the results of our data for the three temperature sensors in each environment.

\begin{table}
    \centering
    \begin{tabular}{|c|c|c|c|}
        \hline
         &  Ambient & Cold & Hot \\
         \hline
         Temperature (°C) & 20.6 & 4.5 & 86.3 \\
         \hline
         Voltage (V) & 1.8428 & 2.5041 & 0.3188 \\
         \hline
         Resistance \((k\Omega)\) & 0.8639 & 1.4849 & 0.1008 \\
         \hline
    \end{tabular}
    \caption{Caption}
    \label{tab:Initial_Values}
\end{table}

The values in table 1 are gathered from parts a-d of the experiment, and alongside Eq. (5), are used to calibrate the thermistor from voltages to output temperature.  

\begin{figure}[H]
    \centering
    \includegraphics[width=0.5\textwidth, angle = -90]{lab2images/labview_plots.jpg}
    \caption{Plotting data on LabVIEW}
    \end{figure}
    
    \subsection{Part 1: Ambient Temperature}
    In Part 1 of the experiment we take measurements of the laboratory room temperature using each of the three sensors: thermocouple, thermistor, and IC TMP36. The data displayed below shows the temperature sensors at work for $1$ minute of sampling at $f_{s} = 1000\; s^{-1}$. This means $\Delta t_{s} = (f_{s}\cdot 60)^{-1} = 1.6667\times10^{-5}\; \text{minutes}$.
    
    \begin{figure}[H]
        \centering
        \includegraphics[width=0.655\textwidth]{lab2images/thermistor_volt_roomtemp_1min_plot.png}
        \caption{Thermistor at ambient temperature}
    \end{figure}
    
    \begin{figure}[H]
        \centering
        \includegraphics[width=0.655\textwidth]{lab2images/ICTMP36_roomtemp_1min_plot.png}
        \caption{IC TMP36 at ambient temperature}
    \end{figure}
    
    \begin{figure}[H]
        \centering
        \includegraphics[width=0.655\textwidth]{lab2images/thermocouple_roomtemp_1min_plot.png}
        \caption{Thermocouple at ambient temperature}
    \end{figure}


    
    There are small spikes occurring around half of the period for the Thermistor's output. This disturbance could have been due to the Thermistor's sensitivity to very small changes in temperature, possibly due to slight movement (increased kinetic energy correlates to increased thermal energy).
    
    The TMP36 data appears similar to a white noise response for the entirety of the $1$ minute period with its large temperature interval. This shows the inaccuracy of the TMP36 when comparing to the other two sensors.
    
    Similar to the small change in the Thermistor's output, the Thermocouple sees small changes as it is decreasing to a steady-state. 

%(Answer Observation Questions from Part 2)
%QUESTIONS THAT GOTTA BE ANSWERED.
    %a) time to reach steady state.  time constant of the sensor.  
   % b) how long to cool to a safe, drinkable temp?
    %c) how long to reach steady state.  is the time constant different than a?
  %  d) how long does it take to cool to ambient temp.  is this time constant different than a? 

Part 1 %For ambient temp

\begin{table}
    \centering
    \begin{tabularx}{\textwidth}{|X|X|X|X|X|X|}
        \hline
        & Mean (°C) & Standard Dev. (°C) & 95\% Confidence Interval & Bias Error (\%) & Percent Error (\%) \\
        \hline
        Thermocouple & 21.6671 & 0.0578 & 0.0005 & - & - \\
        \hline
        IC Sensor & 20.8415 & 0.4722 & 0.0038 & - & -\\
        \hline
        Thermistor & 20.6000 & 0.0653 & 4.9328E-04 & - & - \\
        \hline
    \end{tabularx}
    \caption{Part 1 Values for the ambient temperature measurements}
    \label{tab:Ambient_temp_stats}
\end{table}

At ambient room temperatures, all three sensors reported temperatures within xxx\% of the true temperature, which was 20.8\degree C.  The BLANK SENSOR had the most accurate readings, and the BLANK SENSOR was the most precise.  
%use std for precision, mean for accuracy    
%add more about why these relationships make sense (cant do bc i don't have the data yet) 

Part 2: %sudden changes in temp

The time constant, \(\tau\),  is defined as the time it takes for the system to change from its original state by 63.2\%.  For a first order sensor, Equation xx is used.  The time constant is measured from steady-state before a step input until the sensor reaches steady-state again.  Therefore To will be the initial steady-state temperature and Tf will be the final steady-state temperature after the step input. 
%USE THIS THING >>> % T(t)=(Tf=To)(1-exp(-t/tau))+To then find T at t=tau
At a time T=tau, the equation becomes The equation can be written in LaTeX as: \( T_{\tau} = \pm 0.632|T_f - T_o| + T_o \) (+/- depends on increase/decrease in temp) 

After taking the data for Part 1, the thermistor data only gave ~4.9V for the remainder of the experiments.  It is unknown what happened between Part 1 and Part 2, but all data concerning the thermistor for this section is invalid.  

\begin{table}
    \centering
    \begin{tabular}{cccccc}
         &  Part a&  Part b&  Part c&  Part d& \\
         Initial Temp (To)&  t11&  t12&  t13&  t14& \\
         Final Temp (Tf)&  t21&  t22&  t23&  t24& \\
         Time (t)&  t31&  t32&  t33&  t34& \\
         T at t=tau&  t41&  t42&  t43&  t44& \\
         time constant (Tau)&  t51&  t52&  t53&  & \\
    \end{tabular}
    \caption{Thermocouple Data}
    \label{tab:sudden_temp}
\end{table}

\begin{table}
    \centering
    \begin{tabular}{cccccc}
         &  Part a&  Part b&  Part c&  Part d& \\
         Initial Temp (To)&  t11&  t12&  t13&  t14& \\
         Final Temp (Tf)&  t21&  t22&  t23&  t24& \\
         Time (t)&  t31&  t32&  t33&  t34& \\
         T at t=tau&  t41&  t42&  t43&  t44& \\
         time constant (Tau)&  t51&  t52&  t53&  & \\
    \end{tabular}
    \caption{IC Sensor Data}
    \label{tab:sudden_temp2}
\end{table}


%PUT SOME GRAPHS IN SECTIONS BELOW, Not all of em unless u wanna (up to u) BUT MAKE SURE THEY ONLY FEATURE A LITTLE STEADY STATE BUT MAINLY FOCUS ON THE STEP INPUT (TIME CONSTANT) STUFF BC THATS ALL WE WANT (EXCEPT FOR PART B) and maybe D

\textbf{Part 2a}

As in part 1, the ambient temperature was very similar for all 3 sensors.  The time it took for each sensor to reach steady-state after being inserted in the boiling water took longer for the XXX sensor than it did for the XXX sensor.  The XXX sensor was the closest to the true temperature of the boiling water and the XXX was the least accurate.  % thermocouple has large temp range (1400C) + large error (2.2 C), thermistor has mid range (200C) w/ small error (0.2 C), TMP36 has mid range (200C) + big error (2 C)

\textbf{Part 2b}

For each sensor, the starting temperature was within X \degree C of each other, and each sensor took over X seconds to reach a temperature of X %whatever we said earlier
degrees.  The XX sensor took the shortest, the XX sensor was the longest, but they were still within X seconds of each other.  This data produced a linear graph, which can be attributed to the slow temperature change, rather than a sudden spike.  % maybe add another sentence here about sensors idk?

\textbf{Part 2c}

In cold temperatures, the sensors were within X\% of the true temperature, with the XX being the most accurate, and the XX being the least.  After the sudden temperature change, they were within X\% of the true temperature.  As shown in table XX, during the sudden temperature changes, the time constant was XX, which is \textbf{different / the same} as in part a.  

\textbf{Part 2d}

Once again, the steady-states of the hot water were within X\degree C of each other.  The time it took for the sensors to reach the True temperature of the room ranged from X seconds for the XX sensor to X seconds for the XX sensor.  The time constants for this experiment are (different / the same) as in part a.  This is because %explain why or why not?

\textbf{Part 2e}

When comparing parts a and d, when the sensors are submerged into the water, the reaction is much faster compared to when the sensors are pulled out into the air.  A similar concept arises in part b, when the sensors slowly measure the temperature decrease of the water over time.  Water has a much larger specific heat compared to air, which allows for better retention of the heat, and also allows for a more rapid reaction when introduced to a sudden change in temperature.  

\section{Conclusion}

The first goal of this lab was to learn about three different temperature sensors: Thermocouples, IC Sensors, and Thermistors.  The thermocouple and IC Sensor report voltage changes as a response to temperature changes, whereas the Thermistor uses resistance.  The outputted voltage data was calibrated and converted into usable temperature data through equations given in lectures.  The IC Sensor linearly changed in voltage as the temperature changed linearly.  The thermistor has an inversely proportional relationship because it uses resistance rather than voltage.  Our data for the IC sensor and thermocouple supported this, but the lack of valid data for the thermistor in Part 2 left us unable to quantify this relationship.  

The second goal of this lab was the acquire and process data in the LabVIEW software.  After the lab session, each group member can comfortably recreate this experiment.  This includes the physical circuits and connections, as well as the block diagrams and data collection via LabVIEW.  However, it is unknown what the cause is for the lack of valid thermistor data.  It could be a simple wire becoming loose as someone moved the breadboard, or it could be a software related issue.  

The final goal of this lab was to investigate the transient behavior of a thermistor (first-order sensor).  We were unable to properly complete this task.  For a majority of the Part 2 experiments, the thermistor graphs should have looked like logarithmic curves, with a fast rate of change initally before slowing down towards the end.  \

Disregarding the obvious thermistor errors, the rest of the data satisfied every 95\% confidence interval.  This data showed that the most accurate sensor compared to the true mean was the IC Sensor, with the thermocouple having higher precision.


\newpage
\thispagestyle{empty}  % Clear header/footer
\begin{center}
	\vspace*{\fill}
	{\Huge Appendices}
	\vspace*{\fill}
\end{center}

% Start appendices
\newpage
\begin{appendices}
\pagestyle{fancy}
\renewcommand{\thefigure}{A\arabic{figure}}
\setcounter{figure}{0}

\section*{Appendix: t-Distribution Tables}
\hypertarget{1}{\includegraphics[width=0.95\textwidth]{t_distribution_Table_lecture3.png}}
\end{appendices}

\end{document}
