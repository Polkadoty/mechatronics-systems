\documentclass{article}

%package setup
\usepackage{graphicx}
\usepackage{amsmath}
\usepackage{fancyhdr}
\usepackage[margin=1in]{geometry}
\usepackage{comment}
\usepackage{placeins}
\usepackage{parskip}
\usepackage{subcaption}
\usepackage{appendix}
\usepackage{soul}
\usepackage{comment}
\usepackage[hidelinks]{hyperref}
\usepackage{matlab-prettifier}
\usepackage{minted}
\usepackage{enumitem}
\usepackage{float}
\usepackage{textcomp, gensymb}

\pagestyle{fancy}
\fancyhf{} % Clear header/footer settings
\rhead{\thepage} % Page number on the right in the header
\lhead{ASE375 Lab Report 3} % Your lab report title on the left

\begin{document}

\begin{titlepage}
  \centering
  \includegraphics[width=10cm]{ase-logo-formal.png}  % Adjust the width as needed
  \vspace{1cm}  % Add some vertical space
 
  \Large \textbf{ASE 375 Electromechanical Systems}\\
  \large \textbf{Section 14115}\\
  \vspace{0.5cm}
  \textbf{Monday: 3:00 - 6:00 pm}\\
 
  \vspace{1cm}
 
  \hrule
  \vspace{0.5cm}
 
  \Huge \textbf{Report 3:\\
  Measuring Displacement}\\
  \Huge \textbf{}\\
 
  \vspace{0.5cm}
  \hrule
 
  \vspace{1cm}
 
  \normalsize \textbf{Andrew Doty, Andres Suniaga, Dennis Hom}\\
  \normalsize \textbf{Due Date: 02/12/2024}
 
\end{titlepage}
\newpage

\tableofcontents
\thispagestyle{empty}
\newpage

\section{Introduction}
In this experiment we take a look at (1) Using a linear potentiometer to measure displacement of a small scale wing model with weights attached at different locations, (2) Determining the twist angle from these displacement measurements, and (3) Calculating the shear center, bending stiffness, and torsional stiffness of the wing model. 
\vspace(5mm)

This experiement attempts to accurately model measurements from the linear potentiometer in comparison to the ideal linear relationship from theoretical predictions. From these measurements we can then calculate the wing's shear center location and stiffness.  

\section{Equipment}
The Equipment used in this experiment include:

Linear Potentiometer:  A LP804 Series linear potentiometer linearly varies resistance as a function of the displacement.  Accurate to  

Caliper: Precision instrument used to measure the distance between two opposite sides of an object. It consists of two jaws, one fixed and one movable, that can be adjusted to fit the object being measured. Calipers can provide accurate measurements of length, width, and diameter.

Breadboard: a reusable solderless prototyping board used for building electronic circuits. Components are inserted into interconnected rows and columns of holes, allowing for easy and temporary assembly of circuits for testing and experimentation.

Circuit Components:  various length male-to-male jumper wires, 5V power supply.

DAQ:  Data Aquisition system that digitizes analog information into "bins" for a computer.  The specific DAQ had one unit, the NI 9215. 

Magnet:  A small magnet is used to attach the potentiometer to the wing at various points along the span.  

Wing: 

Test Bench: xy positioning etc...


\section{Procedure}
Physical Circuit:
    %describe circuit n stuff blah blah
Software Setup:
    %add image / block diagram
Potentiometer Calibration:
    Align the potentiometer parallel to the caliper.  Slightly extend the caliper and the potentiometer by the same distance and record the voltage output from LabVIEW.  Repeat this multiple times until the potentiometer is fully extended.  Plot the voltage vs. Caliper distance and 
 Record the outputs from the LabVIEW software for each 
Part 1:

Part 2:

\section{Data Processing}
\subsection*{Variables}
\begin{enumerate}[label = \roman*.]
    \item
\end{enumerate}

\subsection*{Equations}
\begin{enumerate}[label = \Roman*.]
    \item 
\end{enumerate} 

Calibration:
\begin{table}[ht]
\centering
\begin{tabular}{|c|c|c|}
\hline
\textbf{X Position} & \textbf{Voltage Response} & \textbf{Standard Deviation} \\
\hline
$10 mm$ & $0.5706 V$ & 0.0007\\
\hline
$20.09 mm$ & $1.013 V$ & 0.00071 \\
\hline
$29.99 mm$ & $1.82 V$ & 0.00088 \\
\hline
$40.04 mm$ & $2.485 V$ & 0.0009\\
\hline
$50.01 mm$ & $3.023 V$ & 0.0008\\
\hline
$60.09 mm$ & $3.699 V$ & 0.00078\\
\hline
$80.06 mm$ & $4.996 V$ & 0.0004\\
\hline
\end{tabular}
\caption{Table of X Position and Voltage Response}
\label{tab:position_voltage}
\end{table}

C = 100.8 mm

\begin{table}[ht]
\centering
\begin{tabular}{|c|c|c|}
\hline
\textbf{X Position} & \textbf{Voltage Response} & \textbf{Standard Deviation} \\
\hline
$0 mm$ & $2.146 V$ & 0.00093 \\
\hline
$10 mm$ & $2.164 V$ & 0.00099\\
\hline
$20 mm$ & $2.184 V$ & 0.00095\\
\hline
$30 mm$ & $2.200 V$ & 0.0010 \\
\hline
$40 mm$ & $2.227 V$ & 0.00093 \\
\hline
$50 mm$ & $2.256 V$ & 0.0010 \\
\hline
$60 mm$ & $2.279 V$ & 0.00099 \\
\hline
$70 mm$ & $2.306 V$ & 0.00098 \\
\hline
$80 mm$ & $2.334 V$ & 0.00097\\
\hline
$90 mm$ & $2.363 V$ & 0.00100\\
\hline
$100 mm$ & $2.384 V$ & 0.00096\\
\hline
\end{tabular}
\caption{Table of X Position, Voltage Response, and Standard Deviation Trailing Edge}
\label{tab:position_voltage_stddev}
\end{table}

Calipers go to 0.5 mm in least count
250g weight



\begin{table}[ht]
\centering
\begin{tabular}{|c|c|c|}
\hline
\textbf{X Position} & \textbf{Voltage Response} & \textbf{Standard Deviation} \\
\hline
$0 mm$ & $1.979 V$ & 0.00096\\
\hline
$10 mm$ & $1.988 V$ & 0.00096 \\
\hline
$20 mm$ & $1.988 V$ & 0.00096 \\
\hline
$30 mm$ & $1.985 V$ & 0.00095 \\
\hline
$40 mm$ & $1.993 V$ & 0.00096 \\
\hline
$50 mm$ & $1.997 V$ & 0.00100 \\
\hline
$60 mm$ & $1.990 V$ & 0.00099 \\
\hline
$70 mm$ & $2.004 V$ & 0.00097 \\
\hline
$80 mm$ & $2.005 V$ & 0.00099 \\
\hline
$90 mm$ & $2.014 V$ & 0.00100 \\
\hline
$100 mm$ & $2.014 V$ & 0.00098 \\
\hline
\end{tabular}
\caption{Table of X Position, Voltage Response, and Standard Deviation Leading Edge}
\label{tab:position_voltage_stddev2}
\end{table}

\subsection{Part 2}

\begin{table}[ht]
\centering
\begin{tabular}{|c|c|c|c|}
\hline
\textbf{Potential Position} & \textbf{Weight Position} & \textbf{Volts} & \textbf{Std} \\
\hline
$5$ & $6$ & $2.501 V$ & 0.00093 \\
\hline
$2$ & $6$ & $2.234 V$ & 0.00099 \\
\hline
$4$ & $6$ & $2.809 V$ & 0.00095 \\
\hline
$1$ & $6$ & $2.471 V$ & 0.00090 \\
\hline
$3$ & $6$ & $1.988 V$ & 0.00097 \\
\hline
$6$ & $6$ & $2.102 V$ & 0.00100 \\
\hline
$5$ & $5$ & $2.599 V$ & 0.00100 \\
\hline
$2$ & $5$ & $2.312 V$ & 0.00096 \\
\hline
$4$ & $5$ & $2.873 V$ & 0.00096 \\
\hline
$1$ & $5$ & $2.505 V$ & 0.00090 \\
\hline
$4$ & $4$ & $2.904 V$ & 0.00099 \\
\hline
$1$ & $4$ & $2.564 V$ & 0.00098 \\
\hline
\end{tabular}
\caption{Table of Potential Position, Weight Position, Volts, and Std}
\label{tab:position_weight_volts_std}
\end{table}



\section{Results and Analysis}

\section{Conclusion}

\newpage
\thispagestyle{empty}  % Clear header/footer
\begin{center}
	\vspace*{\fill}
	{\Huge Appendices}
	\vspace*{\fill}
\end{center}

% Start appendices
\newpage
\begin{appendices}
\pagestyle{fancy}
\renewcommand{\thefigure}{A\arabic{figure}}
\setcounter{figure}{0}

\section*{Appendix: t-Distribution Tables}
\hypertarget{1}{\includegraphics[width=0.95\textwidth]{t_distribution_Table_lecture3.png}}
\end{appendices}

\end{document}
